%% This is an example first chapter.  You should put chapter/appendix that you
%% write into a separate file, and add a line \include{yourfilename} to
%% main.tex, where `yourfilename.tex' is the name of the chapter/appendix file.
%% You can process specific files by typing their names in at the
%% \files=
%% prompt when you run the file main.tex through LaTeX.
\section{Complexity}
\label{sec:complex}

Problem solving is based on the complexity of a problem and not only a particular algorithm that solves it \cite{sipser2006}.

\begin{defn}
Let $\Sigma$ be a finite alphabet, $\Sigma^*$ every word derived from $\Sigma$, $L \subseteq \Sigma^*$ is a decision problem.
\end{defn}

\begin{defn}
The algorithm $A$ decides problem $L \subseteq \Sigma^*$ if for all word $w \in \Sigma^*$:
\begin{itemize}
  \item $A$ finishes and returns TRUE if $w \in L$.
  \item $A$ finishes and returns FALSE if $w \notin L$.
\end{itemize}
\end{defn}

\begin{defn}
A problem is verifiable if there's an algorithm that verifies it.
\end{defn}

\begin{defn}
A problem is decidable if there's an algorithm that decides it.
\end{defn}

\subsection{P vs NP}

\begin{defn}
A problem $L \in \mathcal{P}$ if $L$ can be decided in polynomial time $\mathcal{O}(n^k)$.
\end{defn}

\begin{defn}
A problem $L \in \mathcal{NP}$ if $L$ can be verified in polynomial time $\mathcal{O}(n^k)$. Thus, $\mathcal{P} \subseteq \mathcal{NP}$.
\end{defn}

\subsection{$\exists \mathbb{R}$ complexity class}

$\exists \mathbb{R}$ is the class that describes the problems that can be reduced to \textit{the existential theory of the reals}\cite{ExistentialTheoryReals2006}. The decidability of the existential theory of the reals is the problem that decides if a sentence of this form is true:

$$(\exists X_1 \dots \exists X_n): F(\exists X_1, \dots,\exists X_n)$$\\

where $F$ is a quantifier-free formula in the reals. In other words, it's a
conjuntion of clauses where each clause is a real polynomial inequality where
each variable $X_k$ is a real number. We can see that ETR is NP-hard because
SAT can be reduced to it.


\begin{proof}
  Let's take an instance of SAT $\phi_{SAT}$ with clauses $c_k$ and variables
  $x_k$, we can construct an instance of ETR $\phi_{ETR}$ where we can
  construct variables in the domain $\{0,1\}$ with this equality, so for each
  variable $X_k$:
  $$X_k - X_k^2 = 0$$

  Each litteral of each clause will be positive or negative depending if the litteral is cancelled in $\phi_{SAT}$:

  $$x_k \to l = X_k$$
  $$\neg x_k \to l = -X_k$$

  Then for each clause we can have a polynomial that will sum the value of every litteral in the clause must be greater that one, so that at least one litteral is true:

  $$\sum_{l\in c_k} l \geq 1$$

  With this proof, it's easy to see that $\phi_{ETR}$ is valid if and only if $\phi_{SAT}$ is also valid.  \qed\\

\end{proof}

This result can show us that $P \subseteq NP \subseteq \exists \mathbb{R}$.

\subsubsection{Problems in $\exists \mathbb{R}$}

In this section we will describe some problems that are $\exists
\mathbb{R}$-complete and will give an overview about the proof since it is
not the main goal of this paper (donner détail de pourquoi je donne un
overview).

\paragraph{The art gallery problem} Given a simple polygon $P$ (without
crossings between every side), we introduce \textit{guards}. A guard $g$ is
a point that every point of the polygon is watched by a guard. A point
$p$ is watched by a point $q$ if the segment $pq$ is contained in $P$. The
subset $G$, being $G$ the set of guards and $G \subseteq P$, is optimum if it
has the minimal cardinality covering the whole polygon.

The art gallery problem decides given a polygon $P$ and a number of guards $k$
if there exists a configuration of $k$ guards in $G$ guarding the whole
polygon. The art gallery problem is $\exists \mathbb{R}$-complete
\cite{abrahamsenArtGalleryProblem2017}.

\paragraph{Proof idea} First of all, we can see that the art gallery problem
is in $\exists \mathbb{R}$ if we reduce this problem to ETR. If we have an
instance $(P,k)$ of the art gallery problem we can have a formula
\cite{EFRAT2006238} like this:

$$\phi = \{\exists x_1y_1,\dots x_ky_k \forall p_xp_y :
\text{INSIDE-POLYGON}(p_x,p_y) \to \bigvee_{1 \leq i \leq k}
\text{SEES}(x_i,y_i,p_x,p_y)\}$$

Where INSIDE-POLYGON returns $1$ if $(p_x,p_y) \in P$ and SEES returns $1$ if
the segment $(x,y)(p_x,p_y) \in P$. $\phi$ is not a ETR formula, so we'd like
to construct a quantifier-free formula with the idea of $\phi$. To achieve this,
the main idea is to have a small set of points $Q \subseteq P$ such that if these
points are watched, the whole polygon is watched. This subset $Q$ is called
the \textit{witness set}. The only thing is now to create a polynome for each
point that ensures the point is watched by a guard.

To finish the proof we have to prove that the art gallery problem is $\exists
\mathbb{R}$-hard. For this part an $\exists \mathbb{R}$-complete
problem has been deducted from ETR. For the problem ETR-INV we have a set of
variables $\{x_1,\dots,x_n\}$ and a set of equations of this form:

$$x = 1,\ \ x + y = z,\ \ x \cdot y = 1 $$

and the problem decides if it exists a solution to this set of equations such
that the value of each variable is real in $[\frac{1}{2},2]$.

A reduction of ETR-INV is found to the art gallery problem by constructing
a polygon $P$ and finding a number $g$ for that polygon such that the instance
of ETR-INT is true if and only if $P$ is covered by at most $g$ guards.

\paragraph{Unit Disk Graph recognition} The Unit Disk Graph recognition is
the problem that decides if a graph $G$ has a realization $\phi$ as a Unit
Disk Graph. Unit Disk Graph recognition is $\exists \mathbb{R}$-complete.

Recognition of Unit Disk Graphs is $\exists \mathbb{R}$-complete. (corollary of graph realizability problem)\cite{Schaefer2013}

Stretchability is $\exists \mathbb{R}$-complete.
