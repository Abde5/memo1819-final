%% This is an example first chapter.  You should put chapter/appendix that you
%% write into a separate file, and add a line \include{yourfilename} to
%% main.tex, where `yourfilename.tex' is the name of the chapter/appendix file.
%% You can process specific files by typing their names in at the
%% \files=
%% prompt when you run the file main.tex through LaTeX.
\section{Complexity}
\label{sec:complex}

Complexity theory has the objective to establish lower bounds on how efficient
an algorithm can be for a given problem \cite{sipser2006}. This approach let us
have a reference point to establish the difficulty of a problem.


\begin{defn}
Let $\Sigma$ be a finite alphabet, $\Sigma^*$ every word derived from $\Sigma$, $L \subseteq \Sigma^*$ is a decision problem.
\end{defn}

\begin{defn}
  A decider for a decision problem $A$ is an deterministic algorithm $V$ where
    $$A = \{w | V \textit{accepts } w\}$$
  $A$ is polynomially decidable if it has a polynomial time decider \cite{sipser2006}.
\end{defn}

\begin{defn}
A verifier for a decision problem $A$ is an deterministic algorithm $V$ where
  $$A = \{w | V \textit{accepts } \langle w,c\rangle \textit{ for some string } c\}$$
$A$ is polynomially verifiable if it has a polynomial time verifier \cite{sipser2006}.
\end{defn}

\subsection{P vs NP}

\begin{defn}
A problem $L \in \mathcal{P}$ if $L$ is polynomially decidable.
\end{defn}

\begin{defn}
A problem $L \in \mathcal{NP}$ if $L$ is polynomially verifiable. Thus, $\mathcal{P} \subseteq \mathcal{NP}$.
\end{defn}

To prove a bound of complexity on an unknown problem $L$ we have to find another
problem with already known complexity and find equivalences between those two. This
can be achieved through \textit{reductions}.

\begin{defn}
  A reduction of a problem $L$ to a problem $M$ is a mapping of an instance of $L$ ($I_L$)
  to an isntance of $M$ ($I_M$) such that $I_L$ is true for the problem $L$ if and
  only if $I_M$ is true for the problem $M$. This is noted $L \leq M$ and $L \leq_P M$
  if the reduction is done in polynomial time.
\end{defn}

With this concept we can define new complexity classes. $\mathcal{NP}$-hard is
the set of problems so that we can reduce every $\mathcal{NP}$ problem to. The set
of problems that are both $\mathcal{NP}$-hard and $\mathcal{NP}$ are called $\mathcal{NP}$-complete.
This is generalized to every complexity class ($\mathcal{P}$, $\exists \mathbb{R}$, RP, etc...)

\paragraph{Satisfiability problem} The satisfiability problem (SAT) is to decide the satisfiability
of a CNF formula $\phi$. A CNF formula is a boolean formula that is a conjunction of multiple
clauses $c_k$. A clause is a disjunction of multiple literals. A literal may be a variable
or a negation of a variable.

\begin{theorem}[Cook-Levin]
  SAT is $\mathcal{NP}$-complete.
\end{theorem}

\paragraph{Clique problem} The clique problem is to find a maximum clique of a graph
$G$.

\begin{theorem}
  CLIQUE is $\mathcal{NP}$-complete. \cite{Karp1972}
\end{theorem}

\begin{theorem}
  CLIQUE is QPTAS when applied to disk graphs. \cite{DBLP:journals/corr/abs-1712-05010}
\end{theorem}

\begin{theorem}[Clark-Colbourn]
  CLIQUE is $\mathcal{P}$ when applied to unit disk graphs. \cite{CLARK1990165}
\end{theorem}

\subsection{$\exists \mathbb{R}$ complexity class}

$\exists \mathbb{R}$ is the class that describes the problems that can be reduced to
\textit{the existential theory of the reals}\cite{ExistentialTheoryReals2006}. The
existential theory of the reals is the problem of deciding if a sentence of this form
is true:

$$(\exists X_1 \dots \exists X_n): F(\exists X_1, \dots,\exists X_n)$$\\

where $F$ is a quantifier-free formula in the reals. In other words, it is a
conjuntion of clauses where each clause is a real polynomial inequality where
each variable $X_k$ is a real number. We can see that ETR is NP-hard because
SAT can be reduced to it.


\begin{proof}
  Let's take an instance of SAT $\phi_{SAT}$ with clauses $c_k$ and variables
  $x_k$, we can construct an instance of ETR $\phi_{ETR}$ where we can
  construct variables in the domain $\{0,1\}$ with this equality, so for each
  variable $X_k$:
  $$X_k - X_k^2 = 0$$

  Each literal of each clause will be positive or negative depending if the literal is cancelled in $\phi_{SAT}$:

  $$x_k \to l = X_k$$
  $$\neg x_k \to l = (1-X_k)$$

  Then for each clause we can have a polynomial for which the sum of the values of every
  literal in the clause must be greater than one, so that at least one literal is true:

  $$\sum_{l\in c_k} l \geq 1$$

  With this proof, it is easy to see that $\phi_{ETR}$ is valid if and only if $\phi_{SAT}$ is also valid.  \qed\\

\end{proof}

This result can show us that $P \subseteq NP \subseteq \exists \mathbb{R}$.

\subsubsection{Problems in $\exists \mathbb{R}$}

In this section we will describe some problems that are $\exists
\mathbb{R}$-complete and will give an overview of the proof.

\paragraph{The art gallery problem} Given a simple polygon $P$ (without
crossings between every side), we introduce \textit{guards}. A guard
$g$ is a point such that every point of the polygon is watched by a guard.
A point $p$ is watched by a point $q$ if the segment $pq$ is contained
in $P$. The subset $G$, being $G$ the set of guards and $G \subseteq
P$, is optimum if it has the minimal cardinality covering the whole
polygon.

The art gallery problem is to decide, given a polygon $P$ and a number of
guards $k$, whether there exists a configuration of $k$ guards in $G$
guarding the whole polygon. The art gallery problem is $\exists
\mathbb{R}$-complete \cite{abrahamsenArtGalleryProblem2017}.

\paragraph{Proof idea} First of all, we can see that the art gallery problem
is in $\exists \mathbb{R}$ if we reduce this problem to ETR. If we have an
instance $(P,k)$ of the art gallery problem we can have a formula
\cite{EFRAT2006238} like this:

$$\phi = \{\exists x_1y_1,\dots x_ky_k \forall p_xp_y :
\text{INSIDE-POLYGON}(p_x,p_y) \to \bigvee_{1 \leq i \leq k}
\text{SEES}(x_i,y_i,p_x,p_y)\}$$

Where INSIDE-POLYGON returns $1$ if $(p_x,p_y) \in P$ and SEES returns $1$ if
the segment $(x,y)(p_x,p_y) \in P$. $\phi$ is not a ETR formula, so we would like
to construct a quantifier-free formula with the idea of $\phi$. To achieve this,
the main idea is to have a small set of points $Q \subseteq P$ such that if these
points are watched, the whole polygon is watched. This subset $Q$ is called
the \textit{witness set}. The only thing is now to create a polynomial for each
point that ensures that the point is watched by a guard.

To finish the proof we have to prove that the art gallery problem is $\exists
\mathbb{R}$-hard. For this part an $\exists \mathbb{R}$-complete
problem has been deducted from ETR. For the problem ETR-INV we have a set of
variables $\{x_1,\dots,x_n\}$ and a set of equations of this form:

$$x = 1,\ \ x + y = z,\ \ x \cdot y = 1 $$

and the problem decides if it exists a solution to this set of equations such
that the value of each variable is real in $[\frac{1}{2},2]$.

A reduction of ETR-INV is found to the art gallery problem by constructing
a polygon $P$ and finding a number $g$ for that polygon such that the instance
of ETR-INT is true if and only if $P$ is covered by at most $g$ guards.

\paragraph{Stretchability} A pseudoline is a simple closed curve in the plane.
Given a pseudoline arrangement, is it equivalent to an arrangement of straight lines?

\paragraph{Proof idea} ETR can be reduced to STRETCHABILITY due to Mnev's
universality theorem. \cite{10.1007/978-3-642-11805-0_32}

\paragraph{Unit disk graph recognition} The unit disk graph recognition is
the problem that decides if a graph $G$ is a unit disk graph. Unit disk graph
recognition is $\exists \mathbb{R}$-complete. \cite{Schaefer2013}

\paragraph{Proof idea} UDG recognition is a corollary of deciding whether a graph
with a given length is realizable. This problem is $\exists \mathbb{R}$-complete.

The reduction is done from STRETCHABILITY \cite{Schaefer2013}. The reduction is
done by adding a vertex to $V$ for each pseudoline intersection. For each three
consecutive points $u_1, u_2, u_3$ along a pseudoline a widget will be added that will
be only realizable if and only if the pseudoline can be stretched with the same
arrangement.
