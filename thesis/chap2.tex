%% This is an example first chapter.  You should put chapter/appendix that you
%% write into a separate file, and add a line \include{yourfilename} to
%% main.tex, where `yourfilename.tex' is the name of the chapter/appendix file.
%% You can process specific files by typing their names in at the
%% \files=
%% prompt when you run the file main.tex through LaTeX.
\section{Complexity}
\label{sec:complex}

Problem solving is based on the complexity of a problem and not only a particular algorithm that solves it \cite{sipser2006}.

\begin{defn}
Let $\Sigma$ be a finite alphabet, $\Sigma^*$ every word derived from $\Sigma$, $L \subseteq \Sigma^*$ is a decision problem.
\end{defn}

\begin{defn}
The algorithm $A$ decides problem $L \subseteq \Sigma^*$ if for all word $w \in \Sigma^*$:
\begin{itemize}
  \item $A$ finishes and returns TRUE if $w \in L$.
  \item $A$ finishes and returns FALSE if $w \notin L$.
\end{itemize}
\end{defn}

\begin{defn}
A problem is verifiable if there's an algorithm that verifies it.
\end{defn}

\begin{defn}
A problem is decidable if there's an algorithm that decides it.
\end{defn}

\subsection{P vs NP}

\begin{defn}
A problem $L \in \mathcal{P}$ if $L$ can be decided in polynomial time $\mathcal{O}(n^k)$.
\end{defn}

\begin{defn}
A problem $L \in \mathcal{NP}$ if $L$ can be verified in polynomial time $\mathcal{O}(n^k)$. Thus, $\mathcal{P} \subseteq \mathcal{NP}$.
\end{defn}

\subsection{$\exists \mathbb{R}$ complexity class}

$\exists \mathbb{R}$ is the class that describes the problems such that they can be reduced to \textit{the existential theory of the reals}\cite{ExistentialTheoryReals2006}. The decidability of the existential theory of the reals is the problem to decide if a sentence of this shape is true:

$$(\exists X_1 \dots \exists X_n): F(\exists X_1, \dots,\exists X_n)$$\\

where $F$ is a quantifier-free formula in the reals. In other words, it's a conjuntion
of clauses where each clause is a real polynomial inequality where each variable $X_k$ is a real number. We can see that ETR is NP-hard because SAT can be reduced to it.


\begin{proof}
  Let's take an instance of SAT $\phi_{SAT}$ with clauses $c_k$ and variables $x_k$, we can construct an instance of ETR $\phi_{ETR}$ where we can construct *booleans* variables with inequalities, so for each variable $X_k$:
  $$0 \leq X_k \leq 1$$
  $$X_k - X_k^2 = 0$$

  Each litteral of each clause will be positive or negative depending if the litteral is cancelled in $\phi_{SAT}$:

  $$x_k \to l = X_k$$
  $$\neg x_k \to l = -X_k$$

  Then for each clause we can have a polynomial that will sum the value of every litteral in the clause must be greater that one, so at least one litteral is true:

  $$\sum_{l\in c_k} l \geq 1$$

  This done, it's easy to see that if $\phi_{ETR}$ is valid if and only if $\phi_{SAT}$ is also valid  \qed\\

\end{proof}

This result can show us that $P \subseteq NP \subseteq \exists \mathbb{R}$
\subsubsection{Problems in $\exists \mathbb{R}$}

The art gallery problem is $\exists \mathbb{R}$-complete.\cite{abrahamsenArtGalleryProblem2017}

Recognition of Unit Disk Graphs is $\exists \mathbb{R}$-complete. (corollary of graph realizability problem)\cite{Schaefer2013}

Stretchability is $\exists \mathbb{R}$-complete.
