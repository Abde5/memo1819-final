%% This is an example first chapter.  You should put chapter/appendix that you
%% write into a separate file, and add a line \include{yourfilename} to
%% main.tex, where `yourfilename.tex' is the name of the chapter/appendix file.
%% You can process specific files by typing their names in at the
%% \files=
%% prompt when you run the file main.tex through LaTeX.
\section{Complexity}

Problem solving is based on the complexity of a problem and not only a particular algorithm that solves it \cite{sipser2006}.

\begin{defn}
Let $\Sigma$ be a finite alphabet, $\Sigma^*$ every word derived from $\Sigma$, $L \subseteq \Sigma^*$ is a decision problem.
\end{defn}

\begin{defn}
The algorithm $A$ decides problem $L \subseteq \Sigma^*$ if for all word $w \in \Sigma^*$:
\begin{itemize}
  \item $A$ finishes and returns TRUE if $w \in L$.
  \item $A$ finishes and returns FALSE if $w \notin L$.
\end{itemize}
\end{defn}

\begin{defn}
A problem is decidable if there's an algorithm that decides it.
\end{defn}

\begin{defn}
A problem is decidable if there's an algorithm that decides it.
\end{defn}

\subsection{P vs NP}

\begin{defn}
A problem $L \in \mathcal{P}$ if $L$ can be decided in polynomial time $\mathcal{O}(n^k)$.
\end{defn}

\begin{defn}
A problem $L \in \mathcal{NP}$ if $L$ can be verified in polynomial time $\mathcal{O}(n^k)$. Thus, $\mathcal{P} \subseteq \mathcal{NP}$.
\end{defn}

\subsection{$\exists \mathbb{R}$ complexity class}

$\exists \mathbb{R}$ is the class that describes the problems such that they can be reduced to \textit{the existential theory of the reals}\cite{ExistentialTheoryReals2006}. The existential theory of the reals

\subsubsection{Problems in $\exists \mathbb{R}$}

The art gallery problem is $\exists \mathbb{R}$-complete.\cite{abrahamsenArtGalleryProblem2017}

Recognition of Unit Disk Graphs is $\exists \mathbb{R}$-complete. (corollary of graph realizability problem)\cite{Schaefer2013}

Stretchability is $\exists \mathbb{R}$-complete.
