\section{Introduction}

Disk graphs are graphs that represent the intersection of disks on the plane. They are
currently used to give a graph-theoretic model to several problems \cite{seiferth2016disk},
mainly in the area of network broadcasting \cite{murphey1999frequency} where the model
is implicit: a set of antennas (senders and receivers) are placed in a terrain and there
has not to be a frequency collision between pretty close antennas to avoid interferences.
This can be modelized by taking a disk for each antenna and their range would be the radius
of its respective disk.\\

The main idea of this thesis is to study a more specific and recently presented class
of disk graphs: Thin Strip Graphs \cite{hayashiThinStripGraphs2017}. The properties
of this class of graphs will be detailed in section \ref{sec:thin}. State of the art and notations
of related topics will be detailed in sections \ref{sec:graphs} (graph theory), \ref{sec:complex}
(complexity theory) and \ref{sec:geom} (geometry). Open questions about this class of graphs
are proposed at the end of the article.
