%% This is an example first chapter.  You should put chapter/appendix that you
%% write into a separate file, and add a line \include{yourfilename} to
%% main.tex, where `yourfilename.tex' is the name of the chapter/appendix file.
%% You can process specific files by typing their names in at the
%% \files=
%% prompt when you run the file main.tex through LaTeX.
\chapter{Graphs and disks}


\section{Graphs}

A graph $G$ is defined as $G(V,E)$, where $V$ is the set of vertices and $E$ the set of edges. A vertex $v \in V$ is the fundamental unit of a graph. An edge $e \in E$ is a structures that links two vertices. The vertices $vw \in V$ that $e \in E$ links are called the \textit{endpoints}.

\begin{definition}
  An embedding of a graph $G$ is a representation of this graph on the plane.
\end{definition}

\begin{definition}
  A graph $G$ is planar iff there is an embedding of this graph that doesn't have any crossing between the edges.
\end{definition}


\section{Intersection graphs}

% This is an example of how you would use tgrind to include an example
% of source code; it is commented out in this template since the code
% example file does not exist.  To use it, you need to remove the '%' on the
% beginning of the line, and insert your own information in the call.
%
%\tagrind[htbp]{./biblio.tex}{Post Multiply Normalization}{opt:pmn}

Given a geometric construction with multiple objects, an intersection graph is a graph that maps the objects into vertices and every intersection between objects is an edge between the corresponding vertices.

\begin{definition}
  A graph $G$ is a comparibility graph if there is a partial order... (check c-strip article, good definition).
\end{definition}

\subsection{Interval graphs}

\subsection{Unit disk graphs}

% This is an example of how you would use tgrind to include an example
% of source code; it is commented out in this template since the code
% example file does not exist.  To use it, you need to remove the '%' on the
% beginning of the line, and insert your own information in the call.
%
%\tgrind[htbp]{code/be.s.tex}{Block Exponent}{opt:be}
