%% This is an example first chapter.  You should put chapter/appendix that you
%% write into a separate file, and add a line \include{yourfilename} to
%% main.tex, where `yourfilename.tex' is the name of the chapter/appendix file.
%% You can process specific files by typing their names in at the
%% \files=
%% prompt when you run the file main.tex through LaTeX.
\section{Graphs and disks}
\label{sec:graphs}


\subsection{Graphs}

A graph $G$ is defined as $G = (V,E)$, where $V$ is the set of vertices and $E$ the set of edges. A vertex $v \in V$ is the fundamental unit of a graph. An edge $e \in E$ links two vertices. The vertices $vw \in V$ that $e \in E$ links are called the \textit{endpoints}.

\begin{defn}
  An embedding of a graph $G$ is a representation of this graph on the plane.
\end{defn}

A graph $G$ is planar if there is an embedding of this graph that doesn't have any crossing between the edges.

\begin{theorem}[Kuratowski]
  A graph $G$ is planar iff it doesn't contain $K_5$ or $K_{3,3}$ as a minor.
\end{theorem}


\subsection{Intersection graphs}


Given a geometric construction with multiple objects, an intersection graph is a
graph that maps the objects into vertices and every intersection between objects
is an edge between the corresponding vertices.

\begin{defn}
  A graph $G$ is a comparibility graph if for each edge $\{u,v\} \in E$ there is
  a partial order $\leq$ such that $u \leq v$ or $v \leq u$.
\end{defn}

\subsubsection{Interval graphs}

Definition of interval Graphs

Properties

Definition of MIXED interval graphs
\subsubsection{Unit disk graphs}


Definition of UDG.

Definition of a realization.
