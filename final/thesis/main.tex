\documentclass[11pt,a4paper,oneside]{book}
\usepackage[hmargin={1.25in,1.25in},vmargin={1.25in,1.25in}]{geometry}

\makeindex
\usepackage{textcomp}
\usepackage{fancyhdr}
\usepackage{makeidx}
\pagestyle{myheadings}
\fancyhf{}
\rhead[\leftmark]{thepage}

\usepackage[T1]{fontenc}
\usepackage{url}

\usepackage[all]{xy}
\usepackage{graphicx}
\usepackage{units}
\usepackage{enumerate}
\usepackage[hidelinks]{hyperref}
\usepackage[T1]{fontenc}


\usepackage{amssymb, amsmath, pict2e}
 \pagestyle{myheadings}

\usepackage{pdfsync}
\usepackage{rotating}
\usepackage{multirow}
\usepackage[normalem]{ulem}
\usepackage{cancel}

\usepackage{color}
\usepackage[usenames,dvipsnames,svgnames,table]{xcolor}
\usepackage{pgf,tikz}
\usetikzlibrary{arrows}
\usepackage{environ}
\usepackage{environ}
\makeatletter
\newsavebox{\measure@tikzpicture}
\NewEnviron{scaletikzpicturetowidth}[1]{%
  \def\tikz@width{#1}%
  \def\tikzscale{1}\begin{lrbox}{\measure@tikzpicture}%
  \BODY
  \end{lrbox}%
  \pgfmathparse{#1/\wd\measure@tikzpicture}%
  \edef\tikzscale{\pgfmathresult}%
  \BODY
}

%%%%%%%%%%%%%%%%  start macros  %%%%%%%%%%%%%%%%%%%%%%%%%%%%%%%%%
\newtheorem{theorem}{Theorem}
\newtheorem{defn}[theorem]{Definition}
\newtheorem{example}[theorem]{Example}
\newtheorem{remark}[theorem]{Remark}
\newtheorem{question}[theorem]{Question}

\newtheorem{lemma}[theorem]{Lemma}
\newtheorem{claim}[theorem]{Claim}
\newtheorem{prop}[theorem]{Proposition}
\newtheorem{corollary}[theorem]{Corollary}
\newtheorem{conjecture}[theorem]{Conjecture}

\newtheorem{hyp}[theorem]{Hypothesis}
\newtheorem{alg}[theorem]{Algorithm}

\newcommand{\qed}{\mbox{$\Box$}}
\newcommand{\proof}{\medbreak\par\noindent{\bf Proof. }}
 \newcommand{\GP}{{\vec{G}}_P}
\newcommand{\GN}{{\vec{G}}_N}

\newcommand{\cover}{\mathrel{\rlap{$\prec$}
                                \rlap{\hskip 0.7em $\cdot$}
                                 \phantom{\prec}}}

\newcommand{\re}{re}

\newcommand{\up}{\mbox{\rm{up}}}
\newcommand{\side}{\mbox{\rm{side}}}
\newcommand{\type}{\mbox{\rm{type}}}


\def\reals{{\mathbb R}}


\newcommand{\ahat}{{\hat{a}}}
\newcommand{\bhat}{{\hat{b}}}
\newcommand{\chat}{{\hat{c}}}
\newcommand{\dhat}{{\hat{d}}}
\newcommand{\bolda}{{\bf{a}}}
\newcommand{\boldb}{{\bf{b}}}
\newcommand{\boldc}{{\bf{c}}}
\newcommand{\boldd}{{\bf{d}}}

\newcommand{\iplus}{{\cal I}^+}
\newcommand{\iminus}{{\cal I}^-}
\newcommand{\ipm}{{\cal I}^{\pm}}
\newcommand{\ipmix}{{\cal I}^{{\cal M}}}

\newcommand{\upl}{{\cal U}^+}
\newcommand{\uminus}{{\cal U}^-}
\newcommand{\upm}{{\cal U}^{\pm}}
\newcommand{\upmix}{{\cal U}^{{\cal M}}}


\newcommand{\ppl}{{\cal P}^+}
\newcommand{\pminus}{{\cal P}^-}
\newcommand{\ppm}{{\cal P}^{\pm}}
\newcommand{\ppmix}{{\cal P}^{{\cal M}}}
\newcommand{\bpm}{{\cal B}^{\pm}}
\newcommand{\tpm}{{\cal T}^{\pm}}
\newcommand{\bpmix}{{\cal B}^{{\cal M}}}
\newcommand{\tpmix}{{\cal T}^{{\cal M}}}
%%%%%%%%%%%%%%%%  end macros  %%%%%%%%%%%%%%%%%%%%%%%%%%%%%%%%%


\parindent0em
\parskip1.5ex

\begin{document}

\frontmatter
\begin{titlepage}
\begin{center}
\textbf{UNIVERSIT\'E LIBRE DE BRUXELLES}\\
\textbf{Facult\'e des Sciences}\\
\textbf{D\'epartement d'Informatique}
\vfill{}\vfill{}

{\Huge  Characterization and complexity of \vspace*{.5cm} \linebreak[4] Thin Strip Graphs}

% s \vspace*{.5cm}  \linebreak[4] possibly on many lines   \linebreak[4]  nevertheless a short title is better

{\Huge \par}
\begin{center}{\LARGE Abdeselam El-Haman Abdeselam}\end{center}{\Huge \par}
\vfill{}\vfill{}
\begin{flushright}{\large \textbf{Promotor :} Prof. Jean Cardinal}\hfill{}{\large Master Thesis in Computer Sciences}\\
{\large }\hfill{}{}\end{flushright}{\large\par}
\vfill{}\vfill{}\enlargethispage{3cm}
\textbf{Academic year 2018~-~2019}
\end{center}
\end{titlepage}
\newpage
\thispagestyle{empty}
\null

\newenvironment{vcenterpage}
{\newpage\thispagestyle{empty}
\vspace*{\fill}}
{\vspace*{\fill}\par\pagebreak}

\begin{vcenterpage}
\begin{flushright}
    \large\em\null\vskip1in
    You may want\\
   to write a dedication here\vfill
  \end{flushright}
\end{vcenterpage}
\thispagestyle{empty}
\vspace*{5cm}

\begin{quotation}
\noindent ``\emph{You may also include one or more general quotes related to your topic.}''
\begin{flushright}\textbf{Name of the author, date}\end{flushright}
\end{quotation}

\medskip

\begin{quotation}
\noindent ``\emph{Another quote.}''
\begin{flushright}\textbf{Name of the author, date}\end{flushright}
\end{quotation}
\chapter*{Acknowledgment}
\thispagestyle{empty}

\noindent I want to thank ...

\thispagestyle{empty}
\setcounter{page}{0}
\tableofcontents
\mainmatter
\setcounter{page}{1}

%% CHAPTERS %%
\chapter*{Conclusions}

This thesis has achieved several observations and results for some problems. We have reviewed related works from several researchers  and regrouped them in a hierarchy of classes. In this hierarchy of classes we added the co-comparability graphs class, which is the supergraph of almost every class of graphs we have seen in this work, except for unit disk graphs. For this, we have proven that the unfettered unit interval graph class is a subset of the co-comparability graphs class.

The most important results in the thesis have been the work on forbidden subgraphs. We have shown that a family of subgraphs of mixed unit interval graphs is also forbidden for unfettered unit interval graphs, which means that this family of forbidden subgraphs is not characteristic of thin strip graphs. For future work, we should research forbidden subgraphs for thin strip graphs that are unit disk graphs and unit interval graphs. This will help us understand the structure of thin strip graphs and give us better results. On the other hand, we have proven that the other forbidden subgraphs for mixed unit interval graphs are in fact thin strip graphs by giving a realization for them.

Finally, the recognition of the graphs that have been looked here have been only overlooked and some observations to tackle unfettered unit interval graphs have been given. The complexity of UUIG can be explored further as this question has not been addressed yet since the definition of Hayashi \textit{et al} \cite{hayashiThinStripGraphs2017}.

A subject that has not been addressed in this thesis is the two-level graphs defined by Breu in his doctoral thesis \cite{breuAlgorithmicAspectsConstrained1996} about constrained disk graphs. An even understanding of thin strip graphs or $c$-strip graphs can be achieved. This class of graphs is close to $c$-strip graphs and is an union of interval graphs \cite{breuAlgorithmicAspectsConstrained1996}. There has not been any comparison between $c$-strip graphs and two-level graphs for the moment in terms of complexity or characterization and some open problems can be questioned for two-level graphs.\\

An exhaustive list of open questions that appeared during the work of this thesis have been compiled in the appendices.

\chapter*{Conclusions}

This thesis has achieved several observations and results for some problems. We have reviewed related works from several researchers  and regrouped them in a hierarchy of classes. In this hierarchy of classes we added the co-comparability graphs class, which is the supergraph of almost every class of graphs we have seen in this work, except for unit disk graphs. For this, we have proven that the unfettered unit interval graph class is a subset of the co-comparability graphs class.

The most important results in the thesis have been the work on forbidden subgraphs. We have shown that a family of subgraphs of mixed unit interval graphs is also forbidden for unfettered unit interval graphs, which means that this family of forbidden subgraphs is not characteristic of thin strip graphs. For future work, we should research forbidden subgraphs for thin strip graphs that are unit disk graphs and unit interval graphs. This will help us understand the structure of thin strip graphs and give us better results. On the other hand, we have proven that the other forbidden subgraphs for mixed unit interval graphs are in fact thin strip graphs by giving a realization for them.

Finally, the recognition of the graphs that have been looked here have been only overlooked and some observations to tackle unfettered unit interval graphs have been given. The complexity of UUIG can be explored further as this question has not been addressed yet since the definition of Hayashi \textit{et al} \cite{hayashiThinStripGraphs2017}.

A subject that has not been addressed in this thesis is the two-level graphs defined by Breu in his doctoral thesis \cite{breuAlgorithmicAspectsConstrained1996} about constrained disk graphs. An even understanding of thin strip graphs or $c$-strip graphs can be achieved. This class of graphs is close to $c$-strip graphs and is an union of interval graphs \cite{breuAlgorithmicAspectsConstrained1996}. There has not been any comparison between $c$-strip graphs and two-level graphs for the moment in terms of complexity or characterization and some open problems can be questioned for two-level graphs.\\

An exhaustive list of open questions that appeared during the work of this thesis have been compiled in the appendices.

%% CHAPTERS %%

\chapter*{Conclusions}

The conclusions are to be written with care\index{Care}, because it will be sometimes the part that could convince a potential reader to read the whole document.

\appendix

\backmatter

\printindex % use makeindex to generate the index

\bibliographystyle{alpha}

\bibliography{main} %use bibtex to generate the bibliography

\end{document}
