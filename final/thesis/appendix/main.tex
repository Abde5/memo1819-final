\begin{appendices}

\chapter{Problems in forbidden induced subgraph characterization}
  \begin{itemize}
    \item \textbf{MUIG}: Joos \cite{joosCharacterizationMixedUnit2013} gives us a complete characterization of forbidden graphs.
    \item \textbf{TSG (Open)}: Hayashi \cite{hayashiThinStripGraphs2017} says that MUIG's forbidden induced subgraphs also are in TSG. He claims that finding a graph $F \in (\text{UDG}\cap\text{UUIG}) \setminus \text{TSG}$ could be a good starting point. In my thesis I show that a forbidden induced subgraph for MUIG is in $\text{UDG}\cap\text{UUIG}$.
    \item \textbf{UDG (Open)}: There is no complete characterization of UDG. This task could not be approachable as it has been proven that the recognition problem for UDG is $\exists\mathbb{R}$-complete.
    \item \textbf{Two-level graph (Open)}: A complete characterization for a two-level graph has not been given yet for every $c$. Breu \cite{breuAlgorithmicAspectsConstrained1996} approached this problem by giving a good basis and several forbidden graphs for this class.
  \end{itemize}

  \chapter{Problems in complexity}
    \begin{itemize}
      \item \textbf{UIG/IG recognition}: Both of these problems are polynomial.
      \item \textbf{MUIG recognition}: Schuchat et al. give a linear algorithm ($\mathcal{O}(|V|^2)$) to recognise MUIGs \cite{shuchatUnitMixedInterval2014}.
      \item \textbf{UDG recognition}: $\exists\mathbb{R}$-complete \cite{ExistentialTheoryReals2006}.
      \item \textbf{SG($c$) recognition (Open)}: Breu  \cite{breuAlgorithmicAspectsConstrained1996} states that SG($c$) recognition is polynomial if a complement edge orientation and a mapping $\phi : V \to [0,c]$ is polynomial as an input of the decision problem. We should research the complexity of the decision problem when the mapping $\phi$ is not given as input.
      \item \textbf{TSG recognition (Open)}: Corollary of previous problem. We should be able to recognize them at least in $\mathcal{NP}$ if the characterization with forbidden subgraphs is complete.
      \item \textbf{UUIG recognition (Open)}: We have seen in this thesis that the problem of UUIG recognition is at least $\mathcal{NP}$. An observation has been shown that could be looked at in further research. This observation has been made while trying to create a polynomial algorithm by looking at maximal cliques. The observation actually proves this intuition wrong.
      \item \textbf{Two-level graph recognition (Open)}: Breu \cite{breuAlgorithmicAspectsConstrained1996} also finds a polynomial algorithm if a mapping $\phi: V \to {0,1}$ that maps a vertex to a level of the graph. It can be an accessible problem as Breu states that a two-level graph is only a union of interval graphs, and these are recognizable in polynomial time.
    \end{itemize}

\end{appendices}
