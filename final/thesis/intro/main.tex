\newtheorem*{_lemma}{Lemma}
\newtheorem*{_obs}{Observation}
\newtheorem*{_theo}{Theorem}

\chapter{Introduction}

\begin{fquote}[Princess Zelda][The Legend of Zelda: Ocarina of Time]
  The flow of time is always cruel... its speed seems different for each person, but no one can change it... A thing that does not change with time is a memory of younger days...
\end{fquote}

This work is mainly focused on the characterization and complexity of variants of unit disk graphs, where the domain of possible locations for the disks is limited. We are also going to see their close relation to a certain family of interval graphs. In this chapter, we will overview the open questions we will focus on and our main results. Further details about the results discussed in this chapter will be introduced later in the thesis as well as a background in Chapter \ref{chap:background}.

\section*{Interval graphs}

In Chapter \ref{chap:interval} we introduce the concept of interval graphs and some of their use cases. An \emph{interval graph} is a graph such that each one of its vertices are closed intervals on the real line and they are adjacent if they overlap; interval graphs such that the length of each interval is unitary is called \emph{unit interval graphs (UIG)}.

Moreover, we introduce two new subclasses of graphs. \emph{Mixed unit interval graphs (MUIG)} \cite{joosCharacterizationMixedUnit2013} can be seen as unit interval graphs but the endpoints of each interval can be open or closed. Another variant are \emph{unfettered unit interval graphs (UUIG)} \cite{hayashiThinStripGraphs2017}, where we can chose whether two touching intervals (so that one of their endpoints are in the same position) are adjacent or not.

Joos describes the class MUIG \cite{joosCharacterizationMixedUnit2013} with a list of graphs that cannot be MUIGs. Also, Hayashi et al. describe the class UUIG with the next theorem.

\begin{_theo}
  A graph is an UUIG if and only if it has a level structure such that each level is a clique.
\end{_theo}

Finally, we take an algorithmic approach to study these classes of graphs. A graph recognition problem for a class of graphs is the problem to guess whether a given graph is of a certain class. The recognition of MUIG is of $\mathcal{O}(n^2)$ \cite{talonCompletionMixedUnit2014} and the recognition of UUIG is only overviewed. For the moment, we know that recognition of UUIG is in $\mathcal{NP}$.

\section*{Strip graphs}

In Chapter \ref{chap:thinDef} we introduce the main class of graphs of this thesis. \emph{Unit disk graphs (UDG)} are intersection graphs of disks on a plane when the diameter of the disks are unitary. \emph{$c$-strip graphs (SG($c$)} \cite{breuAlgorithmicAspectsConstrained1996} is a subclass of UDG, where the center of the disks can only be located between two horizontal lines with a separation of $c$. More formally, for each disk $v$ in the graph $G$, $v_y \in [0,c]$. Breu \cite{breuAlgorithmicAspectsConstrained1996} defined this class of graphs and studied early phases of its charaterization and recognition. However, this is not complete as there is still no answer to the complexity of TSG recognition.


\subsection*{Thin strip graphs}

\emph{Thin strip graphs (TSG)} is a subclass of UDG that that can be defined as the intersection of every SG($c$) with $c > 0$. This is equivalent to say that $TSG = SG(\varepsilon)$ with $\varepsilon$ and arbitrarily small number. Hayashi et al. \ref{hayashiThinStripGraphs2017} present this class of graphs in their work and found some interesting properties about them.

\begin{_theo}
  There is no constant $t$ such that $TSG = SG(t)$.
\end{_theo}

More importantly, TSG is well located in the hierarchy of the graphs seen until now. We know that $MUIG \subsetneq TSG \subsetneq UUIG$. This helps us to find a characterization for TSG because we know that the characterization of MUIG is complete. We also see that every forbidden graph for MUIG is also forbidden in TSG except for one of them.

\section*{Two-level graphs}
\todo[inline]{to add or not? we'll see in the end}
